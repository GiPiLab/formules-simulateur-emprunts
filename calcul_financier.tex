\documentclass[11pt,a4paper]{scrartcl}
\usepackage[utf8]{inputenc}
\usepackage[kerning=true,tracking=true,spacing=true]{microtype}
\usepackage[francais]{babel}
\usepackage[T1]{fontenc}
\usepackage{lmodern}
\usepackage{amsmath}
\usepackage{graphicx}
\usepackage[french,onelanguage,linesnumbered,longend]{algorithm2e}
%\usepackage[margin=1.5cm, includeheadfoot]{geometry}
\usepackage{hyperref}

\hypersetup{colorlinks=true,linkcolor=black,urlcolor=black,
	pdfauthor={Thibault Mondary, Laboratoire de Recherche pour le Développement Local},
	pdftitle={Simulateur d'emprunts, formules et tableaux d'amortissement},
	pdfsubject={formules mathématiques du simulateur d'emprunts}
}



%\addtolength{\parskip}{\baselineskip}

\SetKwRepeat{Struct}{structure \{}{\}}%




\title{Simulateur d'emprunts, formules et tableaux d'amortissement}
\date{\today}
\author{Thibault Mondary\\Laboratoire de Recherche pour le Développement Local\\\small\url{http://gipilab.org}}


\begin{document}
\maketitle

\section{Présentation}

Un emprunt est caractérisé par :

\begin{itemize}
	\item la somme empruntée (notée $sommeEmpruntee$);
	\item le taux d'intérêt annuel, en \% (notée $tauxAnnuel$) ;
	\item la périodicité de remboursement (notée $periodicite$), qui peut être annuelle, semestrielle, trimestrielle ou mensuelle ;
	\item la durée totale de l'emprunt, exprimée nombre d'échéances (notée $duree$) ;
	\item la somme à rembourser à chaque échéance $i$ (notée $paiement_{i}$) ;
\end{itemize}

Connaissant la périodicité de remboursement et trois des quatre grandeurs caractérisant un emprunt, il est possible de calculer la grandeur manquante, ce qui permet de répondre à des questions comme :

\begin{itemize}
	\item Quelle somme vais-je rembourser chaque année si j'emprunte 1\,000\,000 euros à 4,5\% par an pendant 10 ans ? [calcul de l'échéance]
	\item Combien de temps durerait un emprunt de 1\,000\,000 euros à 4,5\% par an en remboursant 126\,378,72 euros tous les ans ? [calcul de la durée]
	\item Combien puis-je emprunter si je choisis de rembourser 126\,378,72 euros par an pendant 10 ans avec un taux d'intérêt de 4,5\% ? [calcul de la somme empruntée ou capital]
	\item Quel est le taux d'intérêt appliqué à un emprunt de 1\,000\,000 euros pendant 10 ans en remboursant 126\,378,72 euros tous les ans ? [calcul du taux annuel]
\end{itemize}



La somme à rembourser à chaque échéance $i$ est composée d'une part de capital (notée $partCapital_{i}$) et d'une part d'intérêts (notée $partInterets_{i}$). La répartition entre ces deux valeurs est déterminée par le \emph{profil d'amortissement}. Un tableau d'amortissement présente ces valeurs de manière synthétique pour chaque échéance, un exemple est présenté sur le tableau \ref{tblAmort1}.


\begin{table}
\centering
\caption{Emprunt de 1\,000\,000 à 4,5\% par an pendant 10 ans}
\begin{tabular}{rrrrrr}
i-eme & $date$ & $paiement$ & $partCapital$ & $partInterets$ & $restant$ \\ 
\hline
1 & 16/09/2015 & 126\,378,82 & 81\,378,82 & 45\,000 & 918\,621,18 \\ 
2 & 16/09/2016 & 126\,378,82 & 85\,040,87 & 41\,337,95 & 833\,580,31 \\ 
3 & 16/09/2017 & 126\,378,82 & 88\,867,71 & 37\,511,11 & 744\,712,6 \\ 
4 & 16/09/2018 & 126\,378,82 & 92\,866,75 & 33\,512,07 & 651\,845,85 \\ 
5 & 16/09/2019 & 126\,378,82 & 97\,045,76 & 29\,333,06 & 554\,800,09 \\ 
6 & 16/09/2020 & 126\,378,82 & 101\,412,82 & 24\,966 & 453\,387,27 \\ 
7 & 16/09/2021 & 126\,378,82 & 105\,976,39 & 20\,402,43 & 347\,410,88 \\ 
8 & 16/09/2022 & 126\,378,82 & 110\,745,33 & 15\,633,49 & 236\,665,54 \\ 
9 & 16/09/2023 & 126\,378,82 & 115\,728,87 & 10\,649,95 & 120\,936,67 \\ 
10 & 16/09/2024 & 126\,378,82 & 120\,936,67 & 5442,15 & 0,00 \\ 
\hline
\multicolumn{1}{l}{Total} &  & 1\,263\,788,22 & 1\,000\,000 & 263\,788,22 & \\ 
\end{tabular}
\label{tblAmort1}
\end{table}



Deux profils d'amortissement usuels sont présentés dans la suite de ce document, le profil à échéances constantes et celui à capital constant. Le profil à échéances constantes impose que la somme remboursée lors de chaque échéance ne varie pas durant la durée de l'emprunt. Le profil d'amortissement à capital constant impose quant à lui que la part de capital contenue dans la somme remboursée lors de chaque échéance reste constante. En conséquence, la valeur remboursée à chaque paiement n'est plus constante.

Le tableau \ref{tblAmort2} présente le même emprunt que le tableau \ref{tblAmort1} mais avec un profil à capital constant. On remarque que si la somme empruntée est la même (1\,000\,000), le coût total de l'emprunt (total des échéances) est inférieur (1\,247\,500 par rapport à 1\,263\,788,22). En contrepartie durant la première moitié de la durée de l'emprunt, les sommes remboursées chaque année sont plus élevées.


\begin{table}
\centering
\caption{Emprunt de 1\,000\,000 à 4,5\% par an pendant 10 ans, profil capital constant}
\begin{tabular}{rrrrrr}
i-eme & $date$ & $paiement$ & $partCapital$ & $partInterets$ & $restant$ \\ 
	\hline
	1 & 16/09/2015 & 145\,000,00& 100\,000,00& 45\,000,00&900\,000,00\\ 
	2 & 16/09/2016 & 140\,500,00& 100\,000,00& 40\,500,00&800\,000,00\\ 
	3 & 16/09/2017 & 136\,000,00& 100\,000,00& 36\,000,00&700\,000,00\\ 
	4 & 16/09/2018 & 131\,500,00& 100\,000,00& 31\,500,00&600\,000,00\\ 
	5 & 16/09/2019 & 127\,000,00& 100\,000,00& 27\,000,00&500\,000,00\\ 
	6 & 16/09/2020 & 122\,500,00& 100\,000,00& 22\,500,00&400\,000,00\\ 
	7 & 16/09/2021 & 118\,000,00& 100\,000,00& 18\,000,00&300\,000,00\\ 
	8 & 16/09/2022 & 113\,500,00& 100\,000,00& 13\,500,00&200\,000,00\\ 
	9 & 16/09/2023 & 109\,000,00& 100\,000,00& 9\,000,00&100\,000,00\\ 
	10& 16/09/2024 & 104\,500,00& 100\,000,00& 4\,500,00& 0,00\\ 
	\hline
	\multicolumn{1}{l}{Total} & \multicolumn{1}{l}{} & 1\,247\,500,00 & 1\,000\,000,00 & 247\,500,00 & \\ 
\end{tabular}
\label{tblAmort2}
\end{table}


\section{Profil d'amortissement à échéances constantes}
Dans la suite du document, la valeur de $periodicite$ est définie comme le nombre de périodes dans une année, ainsi :
\begin{equation}
	periodicite =
	\begin{cases}
		12 & \text{en cas de remboursement mensuel} \\
		4 & \text{en cas de remboursement trimestriel} \\
		2 & \text{en cas de remboursement semestriel} \\
		1 & \text{en cas de remboursement annuel} \\
	\end{cases}
	\label{eqnperiodicite}
\end{equation}

Dans cette section nous présentons les formules permettant de calculer la somme à payer, la somme empruntée, le taux d'intérêts annuel et la durée d'un emprunt dont l'amortissement s'effectue selon le profil à échéances constantes. La valeur de $periodicite$ est supposée connue et est définie sur l'équation \ref{eqnperiodicite}.


\subsection{Calcul de la somme à payer à chaque échéance}
\emph{Quelle somme vais-je rembourser chaque année si j'emprunte 1\,000\,000 euros à 4,5\% par an pendant 10 ans ?}


Soit un emprunt dont la somme empruntée, le taux annuel et la durée sont connus, la somme à payer à chaque échéance vaut :
\begin{equation}
	paiement=\begin{cases}
	\frac{sommeEmpruntee}{duree} & \text{si } tauxAnnuel = 0 \\
	\frac{sommeEmpruntee \times \frac{tauxAnnuel}{periodicite}}{1-(1+\frac{tauxAnnuel}{periodicite})^{-duree}} & \text{si } tauxAnnuel > 0
	\end{cases}
\end{equation}
		

\subsection{Calcul de la somme empruntée}
\emph{Combien puis-je emprunter si je choisis de rembourser 126\,378,72 euros par an pendant 10 ans avec un taux d'intérêt de 4,5\% ?}

Soit un emprunt dont le taux annuel, la durée et la somme à payer à chaque période ($paiement$) sont connus, la somme empruntée vaut :

		\begin{equation}
			sommeEmpruntee=
			\begin{cases}
				\frac{paiement \times ((\frac{tauxAnnuel}{periodicite}+1)^{duree}-1)}{\frac{tauxAnnuel}{periodicite} \times (\frac{tauxAnnuel}{periodicite}+1)^{duree}}&\text{si }tauxAnnuel \neq 0\\
				paiement \times duree&\text{sinon}
			\end{cases}
		\end{equation}

\subsection{Calcul de la durée de l'emprunt}
\emph{Combien de temps durerait un emprunt de 1\,000\,000 euros à 4,5\% par an en remboursant 126\,378,72 euros tous les ans ?}

Soit un emprunt dont la somme empruntée, le taux annuel et la somme à payer lors de chaque échéance sont connus, sa durée vaut (arrondir à l'entier supérieur) :



\begin{equation}
	duree=
	\begin{cases}
		\lceil\frac{log(-(paiement/(\frac{tauxAnnuel}{periodicite}\times sommeEmpruntee-paiement)))}{log(1+\frac{tauxAnnuel}{periodicite})}\rceil& \text{si } tauxAnnuel \neq 0\\
		\lceil\frac{sommeEmpruntee}{paiement}\rceil& \text{sinon}
	\end{cases}
\end{equation} 
	



\subsection{Calcul du taux annuel}
\emph{Quel est le taux d'intérêt appliqué à un emprunt de 1\,000\,000 euros pendant 10 ans en remboursant 126\,378,72 euros tous les ans ?}

Soit un emprunt dont la somme empruntée, la durée et la somme à payer lors de chaque échéance sont connus. Il n'existe pas de formule mathématique pour déterminer directement le taux annuel. Une solution simple pour approximer le taux annuel est de faire varier la valeur du taux, de calculer $paiement$ à l'aide de ce taux, puis de mesurer l'écart obtenu entre le $paiement$ connu et le $paiement$ calculé, comme présenté sur l'algorithme \ref{algoTauxEC}.

\begin{algorithm}
	\KwIn{$paiement,sommeEmpruntee,duree,periodicite$}
	\KwResult{Le taux annuel ou $-1$ en cas de taux non trouvé}
	\BlankLine
	\lIf{$sommeEmpruntee = paiement \times duree$}{\Return $0$}
	\BlankLine
	$maxIterations \leftarrow 1000000$ ; $iteration \leftarrow 0$\;
	\tcc{Taux calculé et pas d'incrémentation}
	$increment \leftarrow 100000$ ; $tauxAnnuel \leftarrow 0,000001$\;
	\tcc{Granularité maxi pour $increment$ et valeur calculée}
	$diviseurMax \leftarrow 0,00001$ ; $paiementCalcule \leftarrow 0$\;
	$sensVariationTaux \leftarrow faux$ ; $compteurRebonds \leftarrow 0$\;
	\BlankLine
	\While{$iteration<maxIterations$ \textbf{\emph{et}} $compteurRebonds<4$ \textbf{\emph{et}} $tauxAnnuel>0$}{
		$iteration \leftarrow iteration + 1$\;
		\tcc{Calcule la somme à payer avec le taux en cours}
		$paiementCalcule \leftarrow calculePaiementEcheanceConstante(sommeEmpruntee,tauxAnnuel,duree,periodicite)$\;
		\tcc{Compare avec la somme à payer attendue}
		\lIf{$paiementCalcule = paiement$}{\Return $tauxAnnuel$}
		\ElseIf{$paiementCalcule < paiement$}{
			$tauxAnnuel \leftarrow tauxAnnuel + increment$\;
			\If{$sensVariationTaux = faux$}{
				
				\lIf{$increment > diviseurMax$}{
					$increment \leftarrow increment \div 10$
					}
					$sensVariationTaux \leftarrow vrai$\;
					$compteurRebonds \leftarrow compteurRebonds + 1$\;
				}
				\lElse{$compteurRebonds \leftarrow 0$}						
			}
			\Else{
			$tauxAnnuel \leftarrow tauxAnnuel - increment$\;

			\If{$sensVariationTaux = vrai$}{
				\lIf{$increment > diviseurMax$}{
					$increment \leftarrow increment \div 10$
				}
				$sensVariationTaux \leftarrow faux$\;
				$compteurRebonds \leftarrow compteurRebonds + 1$\;
			}
			\lElse{$compteurRebonds \leftarrow 0$}						
			} 
		}
		
		\lIf{$tauxAnnuel < 0$\textbf{\emph{ ou }}$iteration = maxIterations$}{\Return $-1$}
		\lElse{\Return $tauxAnnuel$}
\caption{Approximation du taux d'intérêts annuel, échéances constantes}	
\label{algoTauxEC}
\end{algorithm}


\section{Profil d'amortissement à capital constant}
Dans cette section nous présentons les formules permettant de calculer la somme à payer, la somme empruntée, le taux d'intérêts annuel et la durée d'un emprunt dont l'amortissement s'effectue selon le profil à capital constant, c'est-à-dire pour lequel la valeur de $partCapital$ ne change pas. La valeur de $periodicite$ est supposée connue et est définie sur l'équation \ref{eqnperiodicite}.


\subsection{Calcul de la somme à payer à la première échéance}
\emph{Quelle somme vais-je rembourser la première année\footnote{Dans ce profil $paiement$ étant variable les autres échéances se retrouvent en calculant le tableau d'amortissement.} si j'emprunte 1\,000\,000 euros à 4,5\% par an pendant 10 ans ?}


Soit un emprunt dont la somme empruntée, le taux annuel et la durée sont connus, la somme à payer lors de la première échéance vaut :
\begin{equation}
	paiement_1 = sommeEmpruntee \times \frac{tauxAnnuel}{periodicite} + \frac{sommeEmpruntee}{duree}
\end{equation}
		

\subsection{Calcul de la somme empruntée}
\emph{Combien puis-je emprunter si je choisis de rembourser 126\,378,72 euros lors de la première échéance, pendant 10 ans avec un taux d'intérêt de 4,5\% ?}

Soit un emprunt dont le taux annuel, la durée et la somme à payer lors de la première période ($paiement_1$) sont connus, la somme empruntée vaut :

		\begin{equation}
			sommeEmpruntee=\frac{paiement_1 \times duree}{\frac{tauxAnnuel}{periodicite} \times duree + 1}
		\end{equation}

\subsection{Calcul de la durée de l'emprunt}
\emph{Combien de temps durerait un emprunt de 1\,000\,000 euros à 4,5\% par an, en remboursant 126\,378,72 euros la première année ?}

	Soit un emprunt dont la somme empruntée, le taux annuel et la somme à payer lors de la première échéance ($paiement_1$) sont connus, sa durée vaut (arrondir à l'entier supérieur) :


\begin{equation}
	duree=\lceil\frac{sommeEmpruntee}{paiement_1-(sommeEmpruntee \times \frac{tauxAnnuel}{periodicite})}\rceil
\end{equation} 
	



\subsection{Calcul du taux annuel}
\emph{Quel est le taux d'intérêt appliqué à un emprunt de 1\,000\,000 euros pendant 10 ans en remboursant 126\,378,72 euros la première année ?}

Soit un emprunt dont la somme empruntée, la durée et la somme à payer lors de la première période sont connus. Le taux annuel est déterminé par :

\begin{equation}
	tauxAnnuel=\frac{paiement_1 - \frac{sommeEmpruntee}{duree}}{sommeEmpruntee}\times{periodicite}
\end{equation}



\section{Construction des tableaux d'amortissement}
Les tableaux d'amortissement se construisent ligne par ligne. L'algorithme \ref{algoTableauEC} présente la construction d'un tableau pour le profil à échéances constantes, tandis que l'algorithme \ref{algoTableauCC}, très similaire, se focalise sur le profil à capital constant.

\begin{algorithm}
	\KwIn{$sommeEmpruntee, tauxAnnuel, periodicite, duree, dateDebut$}
	\KwResult{Le tableau d'amortissement}
	\tcc{Structure de données représentant une ligne du tableau}
	\Struct{Echeance}{
		$ieme$: entier positif\\
		$date$: date\\
		$paiement$: réel positif\\
		$partCapital$: réel positif\\
		$partInterets$: réel positif\\
		$restantDu$: réel positif\\
	}

	\tcc{Le tableau d'amortissement est une liste d'échéances}

	$tableauAmortissement \leftarrow liste<Echeance>()$\;
	
	\BlankLine
	\tcc{Calcule la valeur précise de l'échéance} 
	$paiement \leftarrow calculePaiementEcheanceConstante(sommeEmpruntee,tauxAnnuel,duree,periodicite)$\;

	$dateEcheance \leftarrow dateDebut$\;
	$tauxPeriodique \leftarrow tauxAnnuel \div periodicite$\;
	$partInterets \leftarrow sommeEmpruntee \times tauxPeriodique$\;
	$partCapital \leftarrow paiement - partInterets$\;
	$restantDu \leftarrow sommeEmpruntee - partCapital$\;

	
	
	\For{$i \leftarrow 1$ \KwTo $duree$}{

		\Switch{$periodicite$}{
			\uCase(//mensuelle){12}{$dateEcheance \leftarrow dateEcheance + 1\ mois$}
			\uCase(//trimestrielle){4}{$dateEcheance \leftarrow dateEcheance + 3\ mois$}
			\uCase(//semestrielle){2}{$dateEcheance \leftarrow dateEcheance + 6\ mois$}
			\uCase(//annuelle){1}{$dateEcheance \leftarrow dateEcheance + 12\ mois$}
		}

		$uneEcheance \leftarrow \{ieme \leftarrow i;date \leftarrow dateEcheance;paiement \leftarrow paiement;partCapital \leftarrow partCapital;partInterets \leftarrow partInterets;restantDu \leftarrow restantDu\}$\;
		\BlankLine
		$tableauAmortissement.ajouter(uneEcheance)$\;
		\BlankLine
		$partInterets \leftarrow restantDu \times tauxPeriodique$\;
		$partCapital \leftarrow paiement - partInterets$\;
		$restantDu \leftarrow restantDu - partCapital$\;
	}
	\Return $tableauAmortissement$\;

\caption{Tableau d'amortissement échéances constantes}	
\label{algoTableauEC}
\end{algorithm}


\begin{algorithm}
	\KwIn{$sommeEmpruntee, tauxAnnuel, periodicite, duree, dateDebut$}
	\KwResult{Le tableau d'amortissement}
	\tcc{Structure de données représentant une ligne du tableau}
	\Struct{Echeance}{
		$ieme$: entier positif\\
		$date$: date\\
		$paiement$: réel positif\\
		$partCapital$: réel positif\\
		$partInterets$: réel positif\\
		$restantDu$: réel positif\\
	}

	\tcc{Le tableau d'amortissement est une liste d'échéances}

	$tableauAmortissement \leftarrow liste<Echeance>()$\;
	
	\BlankLine
	\tcc{Calcule la valeur précise de la première échéance} 
	$paiement \leftarrow calculePaiementCapitalConstant(sommeEmpruntee,tauxAnnuel,duree,periodicite)$\;

	$dateEcheance \leftarrow dateDebut$\;
	$tauxPeriodique \leftarrow tauxAnnuel \div periodicite$\;
	$partInterets \leftarrow sommeEmpruntee \times tauxPeriodique$\;
	$partCapital \leftarrow sommeEmpruntee \div duree$\;
	$restantDu \leftarrow sommeEmpruntee - partCapital$\;

	
	
	\For{$i \leftarrow 1$ \KwTo $duree$}{

		\Switch{$periodicite$}{
			\uCase(//mensuelle){12}{$dateEcheance \leftarrow dateEcheance + 1\ mois$}
			\uCase(//trimestrielle){4}{$dateEcheance \leftarrow dateEcheance + 3\ mois$}
			\uCase(//semestrielle){2}{$dateEcheance \leftarrow dateEcheance + 6\ mois$}
			\uCase(//annuelle){1}{$dateEcheance \leftarrow dateEcheance + 12\ mois$}
		}

		$uneEcheance \leftarrow \{ieme \leftarrow i;date \leftarrow dateEcheance;paiement \leftarrow paiement;partCapital \leftarrow partCapital;partInterets \leftarrow partInterets;restantDu \leftarrow restantDu\}$\;
		\BlankLine
		$tableauAmortissement.ajouter(uneEcheance)$\;
		\BlankLine
		$partInterets \leftarrow restantDu \times tauxPeriodique$\;
		$restantDu \leftarrow restantDu - partCapital$\;
		$paiement \leftarrow partInterets + partCapital$\;
	}
	\Return $tableauAmortissement$\;

\caption{Tableau d'amortissement capital constant}	
\label{algoTableauCC}
\end{algorithm}


\newpage

\vspace*{\fill}

\begin{center}
\includegraphics{by-nc-sa_eu.pdf}

Cette oeuvre du Laboratoire de Recherche pour le Développement Local est mise à disposition selon les termes de la licence Creative Commons Attribution - Pas d’Utilisation Commerciale - Partage dans les Mêmes Conditions 4.0 International\\\url{http://creativecommons.org/licenses/by-nc-sa/4.0/}
\end{center}

\vspace*{\fill}
\end{document}
